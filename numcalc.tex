\documentclass{article}

\usepackage{fontspec}
\usepackage[
	colorlinks=true,%
	anchorcolor=blue,%
	citecolor=blue,%
	linkcolor=blue,%
	urlcolor=blue%
]{hyperref}
\usepackage{polyglossia}
\setdefaultlanguage{brazil}
\usepackage{listings}
\lstset{
	language=C,%
	numbers=left,%
	numberstyle={\color{gray}\scriptsize},%
	numbersep=5pt,%
	showspaces=false
}
\usepackage{xcolor}

\def\pfbox
  {\hbox{\hskip 3pt\lower2pt\vbox{\hrule
  \hbox to 5pt{\vrule height 7pt\hfill\vrule}
  \hrule}}\hskip3pt}

\begin{document}
\title{Noções de Cálculo Numérico}
\author{Adriano J. Holanda}
\date{2022-02-22}
\maketitle

\section*{Localização de Zeros de Funções}

O problema da localização de zeros (raízes) 
de funções contínuas ocorre frequentemente 
em ciência e engenharia~\cite{hamming}.
Iremos mostrar alguns métodos simples 
e fáceis de entender.

\subsection*{Método da bissecção}

O método da bissecção usa o fato do zero de uma 
função contínua $f(x)$ se localizar entre dois 
pontos $x1$ e $x2$ se $f(x1)f(x2)<0$ 
({\it teorema do valor intermediário\/})~\cite{numrec}.

\paragraph{Algoritmo B.}
O algoritmo a seguir usa o método da bissecção para 
localizar o zero de uma função $f(x)$ 
entre os pontos $x_1$ e $x_2$ desde que $f(x_1)f(x_2)<0$ 
com uma acurácia $x_{ac}$ para um número máximo 
de iterações definido por {\tt MAXIT}.

\begin{itemize}
\item[\bf B1] [Inicializar.] Atribuir $i \leftarrow 1$.
\item[\bf B2] [Checar número de iterações.] Se $i>${\tt MAXIT}, 
		ir para B9.

\item[\bf B3] [Achar o ponto médio.] Calcular $x_m = (x_1+x_2)/2$.

\item[\bf B4] [Checar os critérios de parada.] Se $f(x_m)=0$
	(a raiz foi encontrada)
	ou \hbox{$(x_2-x_1)/2 < x_{ac}$} (uma aproximação da
	raiz é suficiente), ir para B9.

\item[\bf B5] [Deslocar um dos limites para o ponto médio.]
	Se $f(x_1)f(x_m)>0$ ir para B6, caso contrário 
	ir para B7.

\item[\bf B6] [Deslocar o ponto médio para $x_1$.] Atribuir
	$x_1\leftarrow x_m$. Ir para o passo B8.

\item[\bf B7] [Deslocar o ponto médio para $x_2$.] Atribuir
	$x_2\leftarrow x_m$.

\item[\bf B8] [Avançar.] Incrementar $i$ em $1$ unidade e
	retornar para B2.

\item[\bf B9] [Término do algoritmo.] Se $i\leq N$, o algoritmo
	terminou com sucesso. Caso contrário, terminou 
	sem sucesso.\quad\pfbox
\end{itemize}

\pagebreak
\paragraph{Programa B.} A seguir uma possível implementação 
do Algoritmo~B.

\lstinputlisting[
	firstline=4,%
	lastline=25,%
	label={lst:zerobis}%
]{src/zerobis.c}

\bibliography{refs}
\bibliographystyle{plain}

\end{document}
