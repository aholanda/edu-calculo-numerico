\documentclass{article}

\usepackage{enumerate}
\usepackage{fontspec}
\usepackage{graphicx}
\usepackage[
	colorlinks=true,%
	anchorcolor=blue,%
	citecolor=blue,%
	linkcolor=blue,%
	urlcolor=blue%
]{hyperref}
\usepackage{polyglossia}
\setdefaultlanguage{brazil}
\usepackage{listings}
\lstset{
	language=C,%
	numbers=left,%
	numberstyle={\color{gray}\scriptsize},%
	numbersep=5pt,%
	showspaces=false
}
\usepackage{xcolor}

\def\pfbox
  {\hbox{\hskip 3pt\lower2pt\vbox{\hrule
  \hbox to 5pt{\vrule height 7pt\hfill\vrule}
  \hrule}}\hskip3pt}

\begin{document}
\title{Noções de Cálculo Numérico}
\author{Adriano J. Holanda}
\date{2022-02-22}
\maketitle

\section*{Localização de Zeros de Funções}

O problema da localização de zeros (raízes) 
de funções contínuas ocorre frequentemente 
em ciência e engenharia~\cite{hamming}.
Iremos mostrar alguns métodos simples 
e fáceis de entender.

\subsection*{Método da bissecção}

O método da bissecção usa o fato do zero de uma 
função contínua $f(x)$ se localizar entre dois 
pontos $x1$ e $x2$ se $f(x1)f(x2)<0$ 
({\it teorema do valor intermediário\/})~\cite{numrec}.

\paragraph{Algoritmo B.}
O algoritmo a seguir usa o método da bissecção para 
localizar o zero de uma função $f(x)$ 
entre os pontos $x_1$ e $x_2$ desde que $f(x_1)f(x_2)<0$ 
com uma acurácia $x_{ac}$ para um número máximo 
de iterações definido por {\tt MAXIT}.

\begin{itemize}
\item[\bf B1] [Inicializar.] Atribuir $i \leftarrow 1$.
\item[\bf B2] [Checar número de iterações.] Se $i>${\tt MAXIT}, 
		ir para B9.

	\item[\bf B3] [Achar o ponto médio.] Calcular $x_m = \frac{x_1+x_2}{2}$.

\item[\bf B4] [Checar os critérios de parada.] Se $f(x_m)=0$
	(a raiz foi encontrada)
		ou \hbox{$|\frac{(x_2-x_1)}{2}| < x_{ac}$} (uma aproximação da
	raiz é suficiente), ir para B9.

\item[\bf B5] [Deslocar um dos limites para o ponto médio.]
	Se $f(x_1)f(x_m)>0$ ir para B6, caso contrário 
	ir para B7.

\item[\bf B6] [Deslocar o ponto médio para $x_1$.] Atribuir
	$x_1\leftarrow x_m$. Ir para o passo B8.

\item[\bf B7] [Deslocar o ponto médio para $x_2$.] Atribuir
	$x_2\leftarrow x_m$.

\item[\bf B8] [Avançar.] Incrementar $i$ em $1$ unidade e
	retornar para B2.

\item[\bf B9] [Término do algoritmo.] Se $i\leq N$, o algoritmo
	terminou com sucesso. Caso contrário, terminou 
	sem sucesso.\quad\pfbox
\end{itemize}

\pagebreak
\paragraph{Programa B.} A seguir uma possível implementação 
do Algoritmo~B.

\lstinputlisting[
	firstline=4,%
	lastline=25,%
	label={lst:zerobis}%
]{src/zerobis.c}

\subsection*{Método da posição falsa}

\paragraph{Algoritmo F.}
O algoritmo a seguir usa o método da posição falsa 
({\it regula falsi\/}) para 
localizar o zero de uma função $f(x)$ 
entre os pontos $x_1$ e $x_2$ desde que $f(x_1)f(x_2)<0$ 
com uma acurácia $x_{ac}$ para um número máximo 
de iterações definido por {\tt MAXIT}.

\begin{itemize}
\item[\bf B1] [Inicializar.] Atribuir $i \leftarrow 1$.
\item[\bf B2] [Checar número de iterações.] Se $i>${\tt MAXIT}, 
		ir para B9.

\item[\bf B3] [Achar a secante da reta que passa por $x_1$ e $x_2$.] 
	Calcular \\\hbox{$x_s = x_1+ f(x_1)\frac{x_2-x_1}{f(x_2)-f(x_1)}$}.

\item[\bf B4] [Checar os critérios de parada.] Se $f(x_m)=0$
	(a raiz foi encontrada)
	ou \hbox{$|x_2-x_1| < x_{ac}$} (uma aproximação da
	raiz é suficiente), ir para B9.

\item[\bf B5] [Deslocar um dos limites para a secante.]
	Se $f(x_s)<0$ ir para B6, caso contrário 
	ir para B7.

\item[\bf B6] [Deslocar $x_1$ para a secante.] Atribuir
	$x_1\leftarrow x_s$. Ir para o passo B8.

\item[\bf B7] [Deslocar $x_2$ para a secante.] Atribuir
	$x_2\leftarrow x_s$.

\item[\bf B8] [Avançar.] Incrementar $i$ em $1$ unidade e
	retornar para B2.

\item[\bf B9] [Término do algoritmo.] Se $i\leq N$, o algoritmo
	terminou com sucesso. Caso contrário, terminou 
	sem sucesso.\quad\pfbox
\end{itemize}

\begin{figure}
	\input zerofal0
	\caption{Aplicação do método da posição falsa para a função $f(x)= \frac{x}{2} ln(x) -1$.}%
\label{pic:latex}%
\end{figure}

\subsection*{Exercícios}


\paragraph{1.} Mudar o valor da acurácia (variável {\tt xac}) de $x$ para $0,1$ e
descrever o que ocorre com o valor da raiz estimada para todos 
os algoritmos. Modifique o valor de {\tt xac} e repita o procedimento.

\paragraph{2.} Para as funções e intervalos a seguir, implemente a localização da raiz
usando os métodos estudados:

\begin{enumerate}[a)]
	\begin{minipage}{0.5\textwidth}
	\item $exp(x) -2x -5\qquad 1\leq x\leq 3$;
	\item $cos(x)-x\qquad 0\leq x\leq\frac{\pi}{2}$; 
	\item $x^3-5\qquad 1\leq x \leq 2,5$;
	\end{minipage}
	\begin{minipage}{0.5\textwidth}
	\item $xln(x)-x-1\qquad 2\leq x\leq 5 $;
	\item $xtan(x)-\frac{1}{x^3}\qquad 2\leq x\leq 4$;
	\item $x^2-7\qquad 1,5\leq x\leq 4$.
	\end{minipage}
\end{enumerate}

\bibliography{refs}
\bibliographystyle{plain}
\end{document}
