
\title{Localização de Zeros de Funções}
\frame{\maketitle}

% \frame{\tableofcontents}

\only<article>{O problema da localização de zeros (raízes) 
de funções contínuas ocorre frequentemente 
em ciência e engenharia~\cite{hamming}.
Iremos mostrar alguns métodos simples 
e fáceis de entender.}

\def\sectiontitle{Método da bissecção}
\section{\sectiontitle}


\begin{frame}{\sectiontitle}
O método da bissecção usa o fato do zero de uma 
função contínua $f(x)$ se localizar entre dois 
pontos $a$ e $b$ se $f(a)f(b)<0$ 
({\it teorema do valor intermediário\/})~\cite{Press1992}.
\end{frame}


\begin{frame}{\sectiontitle}{Algoritmo B.}\footnotesize
O algoritmo a seguir usa o método da bissecção para 
localizar o zero de uma função $f(x)$ 
entre os pontos $a$ e $b$ desde que $f(a)f(b)<0$ 
com uma acurácia $\epsilon$ e um número máximo 
de iterações {\tt MAXIT}.

\begin{itemize}[<+-| alert@+>]
\item[\bf B1] [Inicializar.] Atribuir $i \leftarrow 1$.
\item[\bf B2] [Checar número de iterações.] Se $i>${\tt MAXIT}, 
		ir para B9.

	\item[\bf B3] [Achar o ponto médio.] Calcular $m = \frac{a+b}{2}$.

\item[\bf B4] [Checar os critérios de parada.] Se $f(m)=0$
	(a raiz foi encontrada)
		ou \hbox{$|\frac{(b-a)}{2}| < \epsilon$} (uma aproximação da
	raiz é suficiente), ir para B9.

\item[\bf B5] [Deslocar um dos limites para o ponto médio.]
	Se $f(a)f(m)>0$ ir para B6, caso contrário 
	ir para B7.

\item[\bf B6] [Deslocar o ponto médio para $a$.] Atribuir
	$a\leftarrow m$. Ir para o passo B8.

\item[\bf B7] [Deslocar o ponto médio para $b$.] Atribuir
	$b\leftarrow m$.

\item[\bf B8] [Avançar.] Incrementar $i$ em $1$ unidade e
	retornar para B2.

\item[\bf B9] [Término do algoritmo.] Se $i\leq MAXIT$, o algoritmo
	terminou com sucesso. Caso contrário, terminou 
	sem sucesso.\quad\pfbox
\end{itemize}
\end{frame}


\begin{frame}[fragile]{\sectiontitle}{Programa B.} 

Uma possível implementação do Algoritmo~B.

\lstinputlisting[
	firstline=4,%
	lastline=19,%
	label={lst:bisection}%
]{numcalc/zero/bisection.py}

\end{frame}


\def\sectiontitle{Método da posição falsa}
\section{\sectiontitle}

\begin{frame}{\sectiontitle}

No método da posição falsa ({\it regula falsi\/}), 
são tomados dois pontos $\{a,b\}$ em $x$ que satisfaçam 
a condição $f(a)f(b)<0 \land f(a)<f(b)$, traçando 
uma reta entre estes dois pontos e aproximando a 
secante até a raiz de $f(x)$. A reta entre $a$ 
e $b$ é dada por

\begin{equation}
	\ell(x) = f(a) + \frac{f(b)-f(a)}{b-a}(x-a),
\end{equation}

\noindent cujo zero é dado por

\begin{equation}
	a + f(a)\frac{b-a}{f(b)-f(a)}.
\end{equation}

\end{frame}

\begin{frame}{\sectiontitle}{Algoritmo R.}\footnotesize
O algoritmo a seguir usa o método da posição falsa 
({\it regula falsi\/}) para 
localizar o zero de uma função $f(x)$ 
entre os pontos $a$ e $b$ desde que $f(a)f(b)<0$ 
 com uma acurácia $\epsilon$ para um número máximo 
de iterações $MAXIT$. Para calcular a secante no passo
B3, a condição $f(a)<f(b)$ deve ser satisfeita, caso
contrário, trocar $a$ por $b$ e $b$ por $a$.
 
 \begin{itemize}[<+-| alert@+>]

	 \item[\bf R1] [Inicializar.] Atribuir $i \leftarrow 1$.


	 \item[\bf R2] [Checar número de iterações.] Se $i>${\tt MAXIT}, 
 		ir para R9.

	\item[\bf R3] [Achar a secante da reta que passa por $a$ e $b$.] 
		Calcular \\\hbox{$m = a+ f(a)\frac{b-a}{f(b)-f(a)}$}.

	\item[\bf R4] [Checar os critérios de parada.] Se $f(m)=0$
		(a raiz foi encontrada)
		ou \hbox{$|b-a| < \epsilon$} (uma aproximação da
		raiz é suficiente), ir para R9.

	\item[\bf R5] [Deslocar um dos limites para a secante.]
		Se $f(m)<0$ ir para R6, caso contrário 
		ir para R7.

	\item[\bf R6] [Deslocar $a$ para a secante.] Atribuir
		$a\leftarrow m$. Ir para o passo R8.

	\item[\bf R7] [Deslocar $b$ para a secante.] Atribuir
		$b\leftarrow m$.
 
	\item[\bf R8] [Avançar.] Incrementar $i$ em $1$ unidade e
		retornar para R2.
 
	\item[\bf R9] [Término do algoritmo.] Se $i\leq MAXIT$, o algoritmo
 		terminou com sucesso. Caso contrário, terminou 
	 	sem sucesso.\quad\pfbox
 \end{itemize}

\end{frame}


\begin{frame}[fragile]{\sectiontitle}{Programa R.} 
 Uma possível implementação do Algoritmo~R:
 
 \lstinputlisting[
	firstline=3,%
	lastline=24,%
	basicstyle={\scriptsize},%
	label={lst:regula}%
]{numcalc/zero/regula.py}
\end{frame}

\ifnum1=2

\def\sectiontitle{Método de Newton-Raphson}
\section{\sectiontitle}

\begin{frame}{\sectiontitle}{Algoritmo N}\footnotesize

O algoritmo 
para localizar o zero de uma função $f(x)$ 
entre os pontos $a$ e $b$
com uma acurácia $\epsilon$ para um número máximo 
de iterações $MAXIT$. 

\begin{itemize}[<+-| alert@+>]
\item[\bf N1] [Inicializar.] Atribuir $i \leftarrow 1$ e 
	$m\leftarrow\frac{(a+b)}{2}$.
\item[\bf N2] [Checar número de iterações.] Se $i>${\tt MAXIT}, 
		ir para N9.

	\item[\bf N3] [Achar a distância entre a função $f(x)$ e sua derivada no ponto médio $m$.]
		Calcular $dx\leftarrow\frac{f(m)}{df(m)/dt}$.

	\item[\bf N4] [Atualizar o ponto médio.]
	Calcular $m\leftarrow m-dx$.

\item[\bf N5] [Checar os critérios de parada.] Se 
	\hbox{$|dx| < \epsilon$} (uma aproximação da
	raiz é suficiente), ir para N7.

\item[\bf N6] [Avançar.] Incrementar $i$ em $1$ unidade e
	retornar para N2.

\item[\bf N7] [Término do algoritmo.] Se $i\leq N$, o algoritmo
	terminou com sucesso. Caso contrário, terminou 
	sem sucesso.\quad\pfbox
\end{itemize}

\end{frame}


\begin{frame}[fragile]{\sectiontitle}{Programa N.} 
Uma possível implementação do Algoritmo~N:

\lstinputlisting[
	firstline=3,%
	lastline=17,%
	label={lst:newton}%
]{numcalc/zero/newton.py}

\end{frame}

\fi


\begin{frame}{Referência}

\begin{thebibliography}{Press1992}
	\bibitem[Press1992]{Press1992}
	William H.\ Press {\it et al.}
	\newblock {Numerical Recipes in C (2nd Ed.): The Art of Scientific Computing}.
	\newblock Cambridge University Press, 1992.
\end{thebibliography}
\end{frame}
