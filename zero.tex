\section*{Localização de Zeros de Funções}

O problema da localização de zeros (raízes) 
de funções contínuas ocorre frequentemente 
em ciência e engenharia~\cite{hamming}.
Iremos mostrar alguns métodos simples 
e fáceis de entender.

\subsection*{Método da bissecção}

O método da bissecção usa o fato do zero de uma 
função contínua $f(x)$ se localizar entre dois 
pontos $a$ e $b$ se $f(a)f(b)<0$ 
({\it teorema do valor intermediário\/})~\cite{numrec}.

\paragraph{Algoritmo B.}
O algoritmo a seguir usa o método da bissecção para 
localizar o zero de uma função $f(x)$ 
entre os pontos $a$ e $b$ desde que $f(a)f(b)<0$ 
com uma acurácia $\epsilon$ e um número máximo 
de iterações {\tt N}.

\begin{itemize}
\item[\bf B1] [Inicializar.] Atribuir $i \leftarrow 1$.
\item[\bf B2] [Checar número de iterações.] Se $i>${\tt N}, 
		ir para B9.

	\item[\bf B3] [Achar o ponto médio.] Calcular $m = \frac{a+b}{2}$.

\item[\bf B4] [Checar os critérios de parada.] Se $f(m)=0$
	(a raiz foi encontrada)
		ou \hbox{$|\frac{(b-a)}{2}| < \epsilon$} (uma aproximação da
	raiz é suficiente), ir para B9.

\item[\bf B5] [Deslocar um dos limites para o ponto médio.]
	Se $f(a)f(m)>0$ ir para B6, caso contrário 
	ir para B7.

\item[\bf B6] [Deslocar o ponto médio para $a$.] Atribuir
	$a\leftarrow m$. Ir para o passo B8.

\item[\bf B7] [Deslocar o ponto médio para $b$.] Atribuir
	$b\leftarrow m$.

\item[\bf B8] [Avançar.] Incrementar $i$ em $1$ unidade e
	retornar para B2.

\item[\bf B9] [Término do algoritmo.] Se $i\leq N$, o algoritmo
	terminou com sucesso. Caso contrário, terminou 
	sem sucesso.\quad\pfbox
\end{itemize}

\pagebreak

\paragraph{Programa B.} A seguir uma possível implementação 
do Algoritmo~B.

\lstinputlisting[
	firstline=4,%
	lastline=25,%
	label={lst:bisection}%
]{src/bisection.c}

\subsection*{Método da posição falsa}

No método da posição falsa ({\it regula falsi\/}), 
são tomados dois pontos $\{a,b\}$ em $x$ que satisfaçam 
a condição $f(a)f(b)<0 \land f(a)<f(b)$, traçando 
uma reta entre estes dois pontos e aproximando a 
secante até a raiz de $f(x)$. A reta entre $a$ 
e $b$ é dada por

\begin{equation}
	\ell(x) = f(a) + \frac{f(b)-f(a)}{b-a}(x-a),
\end{equation}

\noindent cujo zero é dado por

\begin{equation}
	\ell(x) = a + f(a)\frac{b-a}{f(b)-f(a)}.
\end{equation}


\section{Método de Newton-Raphson}

\paragraph{Algoritmo N.}
O algoritmo a seguir usa 
o método de Newton-Raphson
para localizar o zero de uma função $f(x)$ 
entre os pontos $a$ e $b$
com uma acurácia $\epsilon$ para um número máximo 
de iterações $N$. 

\begin{itemize}
\item[\bf N1] [Inicializar.] Atribuir $i \leftarrow 1$ e 
	$m\leftarrow\frac{(a+b)}{2}$.
\item[\bf N2] [Checar número de iterações.] Se $i>${\tt MAXIT}, 
		ir para B9.

	\item[\bf N3] [Achar a distância entre a função $f(x)$ e sua derivada no ponto médio $m$.]
		Calcular $dx\leftarrow\frac{f(m)}{df(m)/dt}$.

	\item[\bf N4] [Atualizar o ponto médio.]
	Calcular $m\leftarrow m-dx$.

\item[\bf N5] [Checar os critérios de parada.] Se 
	\hbox{$|dx| < \epsilon$} (uma aproximação da
	raiz é suficiente), ir para N7.

\item[\bf N6] [Avançar.] Incrementar $i$ em $1$ unidade e
	retornar para N2.

\item[\bf N7] [Término do algoritmo.] Se $i\leq N$, o algoritmo
	terminou com sucesso. Caso contrário, terminou 
	sem sucesso.\quad\pfbox
\end{itemize}

\pagebreak

\paragraph{Programa N.} A seguir uma possível implementação 
do Algoritmo~N.

\paragraph{Algoritmo N.}
\lstinputlisting[
	firstline=2,%
	lastline=20,%
	label={lst:newton}%
]{src/newton.c}

\subsection*{Exercícios}

\paragraph{1.} Mudar o valor da acurácia (variável {\tt eps}) de $x$ para $0,1$ e
descrever o que ocorre com o valor da raiz estimada para todos 
os algoritmos. Modifique o valor de {\tt eps} e repita o procedimento.

\paragraph{2.} Para as funções e intervalos a seguir, implemente a localização da raiz
usando os métodos estudados:

\begin{enumerate}[a)]
	\begin{minipage}{0.5\textwidth}
	\item $exp(x) -2x -5\qquad 1\leq x\leq 3$;
	\item $cos(x)-x\qquad 0\leq x\leq\frac{\pi}{2}$; 
	\item $x^3-5\qquad 1\leq x \leq 2,5$;
	\end{minipage}
	\begin{minipage}{0.5\textwidth}
	\item $xln(x)-x-1\qquad 2\leq x\leq 5 $;
	\item $xtan(x)-\frac{1}{x^3}\qquad 2\leq x\leq 4$;
	\item $x^2-7\qquad 1,5\leq x\leq 4$.
	\end{minipage}
\end{enumerate}

\paragraph{3.} Usar o método de {\bf Newton-Raphson} para achar
a raiz da função 

\begin{equation}
	f(x) = cos(x) - x^2 - \frac{x}{5}\qquad para x>0.
\end{equation}
