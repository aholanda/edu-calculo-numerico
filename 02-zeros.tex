\chapter{Localização de Zeros de Funções}


\section{Método da bissecção}

\paragraph{Algoritmo B.} O algoritmo a seguir calcula 
a raiz de uma função $f(x)$ entre os pontos $x_1$ e $x_2$ 
desde que $f(x_1)f(x_2)<0$ com uma acurácia $x_{ac}$
para um número $N$ de iterações.

\begin{enumerate}[1]
	\item[B1] [Inicializar.] Atribuir $i \leftarrow 1$.
	\item[B2] [Achar o ponto médio.] Atribuir $i\leftarrow i+1$ e
		   calcular $x_m = (x_1+x_2)/2$.
	\item[B3] [Checar os critérios de parada.] Se $f(x_m)=0$ 
		(a raiz foi encontrada)
		ou \hbox{$(x_2-x_1)/2 < x_{ac}$} (uma aproximação da 
		raiz é suficiente), ir para B8.
	\item[B4] [Checar o número de iterações.] Se $i > N$, 
		ir para B9.
	\item[B5] [Deslocar um dos limites para o ponto médio.] 
		Se $f(x_1)f(x_m)>0$ ir para B6, caso contrário 
		ir para B7.
	\item[B6] [Deslocar o ponto médio para $x_1$.] Atribuir
		$x_1\leftarrow x_m$. Voltar para o passo B2.
	\item[B7] [Deslocar o ponto médio para $x_2$.] Atribuir
		$x_2\leftarrow x_m$. Voltar para o passo B2.
	\item[B8] [A raiz de $f(x)$ foi encontrada] $r \leftarrow x_m$.
	\item[B9] [A raiz de $f(x)$ não foi encontrada.]
\end{enumerate}

A Listagem~\ref{lst:bisect} mostra uma possível implementação 
do Algoritmo~B usando Python3.

\lstinputlisting[
	lastline=21,%
	label={lst:bisect},%
	caption={Implementação do Algoritmo B (método da bissecção).}
]{numcalc/zeros/bisect.py}
