\subsection*{Exercícios}

\paragraph{1.} Mudar o valor da acurácia (variável {\tt eps}) de $x$ para $0,1$ e
descrever o que ocorre com o valor da raiz estimada para todos 
os algoritmos. Modifique o valor de {\tt eps} e repita o procedimento.

\paragraph{2.} Para as funções e intervalos a seguir, implemente a localização da raiz
usando os métodos estudados:

\begin{enumerate}[a)]
	\begin{minipage}{0.5\textwidth}
	\item $exp(x) -2x -5\qquad 1\leq x\leq 3$;
	\item $cos(x)-x\qquad 0\leq x\leq\frac{\pi}{2}$; 
	\item $x^3-5\qquad 1\leq x \leq 2,5$;
	\end{minipage}
	\begin{minipage}{0.5\textwidth}
	\item $xln(x)-x-1\qquad 2\leq x\leq 5 $;
	\item $xtan(x)-\frac{1}{x^3}\qquad 2\leq x\leq 4$;
	\item $x^2-7\qquad 1,5\leq x\leq 4$.
	\end{minipage}
\end{enumerate}

\paragraph{3.} Usar o método de {\bf Newton-Raphson} para achar
a raiz da função 

\begin{equation}
	f(x) = cos(x) - x^2 - \frac{x}{5}\qquad para x>0.
\end{equation}
